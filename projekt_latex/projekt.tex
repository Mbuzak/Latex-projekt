\documentclass{article}
\usepackage{graphicx}
\usepackage{polski}
\usepackage{multirow}
\usepackage{fancyhdr}
\title{Projekt}
\author{Marian Buzak}
\date{09.01.2020}


\begin{document}
\maketitle 
\pagestyle{fancy}
\fancyhead[L]{}
\begin{center}
Czyli zestawienie losowych informacji
\includegraphics[angle=15,width=250px]{404.png}
\end{center}
\newpage
\tableofcontents
\listoffigures 
\listoftables 
\newpage
\section{Trochę tekstu}
\subsection{Ciekawostki}
Drozofile żyją 2 tygodnie. \\ Młode płetwale błękitne jedzą 4 kilogramy pożywiena na godzinę. \\ Krety potrafią wykopać 20m tunel pod ziemią w ciągu dnia. \\ Planetę Wenus można zobaczyc 3 godziny przed wschodem słońca.
\subsection{I tabela do tego}
\begin{table}[h]
\caption{Gady Polski}
\begin{center}
\begin{tabular}{|c|c|l|}
\hline Rząd & Rodzina & Gatunek\\ \hline \hline
żółwie & żółwie błotne & żółw błotny\\
 &  & żółw ozdobny\\ \hline
łuskonośne & jaszczurki własciwe & jaszczurka zwinka\\
 &  & jaszczurka zielona\\
 &  & jaszczurka żyworodna\\
 &  & jaszczurka murowa\\ \cline{2-3}
 & padalcowate & zaskroniec zwyczajny\\
 & & zaskroniec rybołów\\
 & & gniewosz plamisty\\
 & & wąż Eskulapa\\ \cline{2-3}
 & żmijowate  & żmija zygzakowata\\
 
\hline
\end{tabular}
\end{center}
\end{table}

\begin{figure}[h]
\caption{Latech}
\begin{center}
\includegraphics[width=150pt]{latex.png}
\end{center}
\end{figure}

\section{Jakieś wzory matematyczne}
\subsection{Zbieżność ciągu}
\small{
$$
\sum_{k=1}^n \frac{k!}{2^k}
= 0
$$}
\subsection{Granica wyrażenia}
$$
\lim_{x \to 1}
\frac{3x^5-x^3-2x}{x^5-x}=
\frac{5}{2}
$$

\section{Zależności}
\subsection{Zagadka}
\begin{itemize}
  \item \emph{Było raz tak:}
  \begin{enumerate}
    \item Bociana dziobał szpak
    \item Później była zmiana
    \item I szpak dziobał bociana
    \item Później były jeszcze takie 3 zmiany
  \end{enumerate}
  \item \textbf{\texttt{Ile razy szpak był dziobany?}}
\end{itemize}
\begin{table}[h]
\caption{Ciąg Fibonacciego}
\begin{center}
\begin{tabular}{|c|c|c|}
\hline
5 & 8 & 13\\
\hline
\multicolumn{2}{|c|}{21}&44\\
\hline
\multirow{2}{80px}{65} &109&174 \\
\cline{2-3}
& \multicolumn{2}{|l|}{283}\\
\hline
\end{tabular}
\end{center}
\end{table}
\begin{figure}[h]
\caption{Error}
\begin{center}
\includegraphics[width=125pt]{xp.jpg}
\end{center}
\end{figure}

\newpage
\begin{thebibliography}{9}
\bibitem{lamport94}
 Leslie Lamport,
 \emph{\LaTeX: A Document Preparation System}.
 Addison Wesley, Massachusetts,
 2nd Edition,
 1994. \\
 Partl~\cite{pa} przedstawi\l \ldots
\bibitem{pa} H.~Partl:
\emph{German \TeX},
TUGboat Vol.~9,, No.~1 ('88)
\end{thebibliography}
\end{document}
